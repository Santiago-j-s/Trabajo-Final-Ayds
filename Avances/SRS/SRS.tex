%% LyX 2.1.2 created this file.  For more info, see http://www.lyx.org/.
%% Do not edit unless you really know what you are doing.
\documentclass[english,spanish]{article}
\usepackage[utf8]{luainputenc}
\usepackage{array}
\usepackage{longtable}
\usepackage{calc}
\usepackage{graphicx}

\makeatletter

%%%%%%%%%%%%%%%%%%%%%%%%%%%%%% LyX specific LaTeX commands.
%% Because html converters don't know tabularnewline
\providecommand{\tabularnewline}{\\}

%%%%%%%%%%%%%%%%%%%%%%%%%%%%%% User specified LaTeX commands.
\date{}

\makeatother

\usepackage{babel}
\addto\shorthandsspanish{\spanishdeactivate{~<>}}

\begin{document}

\title{Especificación de Requerimientos del Software}


\author{Dibez Pablo, Santana Santiago}

\maketitle

\section{Introducción}

En el presente documento se explicarán y analizarán los requerimientos
del proyecto \flqq{}\emph{Sistema de Gestión de Pizzería}\frqq{}.
Se adopta el estándar IEEE 830\nocite{1998}.


\subsection{Próposito}

El objetivo del siguiente documento es definir de manera clara y precisa
las funcionalidades y restricciones que tendrá el sistema, a desarrollarse,
de gestión de una pizzeria, va dirigido al equipo de desarrollo del
software y a los usuarios finales del sistema.

Este documento será un medio de comunicación entre todos los roles
implicados en el desarrollo del sistema y por tanto esta sujeto a
revisiones, tanto de los desarrolladores como de los usuarios.


\subsection{Alcance}

Este sistema se encargará de facilitar las operaciones realizadas
en el ciclo habitual de la pizzeria; abarcando el registro de usuarios,
la toma de pedidos, la gestion de hornos y el servicio de distribucion
del producto. El sistema será denominado \emph{Sistema de Gestión
de Pizzería.}


\subsection{Definiciones, acrónimos y abreviaturas}
\begin{itemize}
\item Kit: Conjunto de ingredientes que conforman una determinada variedad
de pizza o empanada
\item Pedido: Solicitud de un producto que un cliente realiza a la pizzería.
\item Usuario: Persona que utiliza algún o algunos subsistemas del \emph{Sistema
de Gestión de Pizzería.}
\item Cliente: Persona que realiza pedidos a la pizzería y paga por ellos.
\end{itemize}

\subsection{Referencias}

\renewcommand\refname{}

\bibliographystyle{ieeetr}
\nocite{*}
\bibliography{Citations}



\subsection{Visión general del documento}

Este documento está conformado de tres secciones que son la Introducción,
la Descripción Global y los Requisitos Específicos. En esta primera
sección se procura proporcionar una visión general de lo que es el
documento de especificación de requisitos. En la segunda sección se
da una descripción general del sistema a construir, para conocer sus
funciones principales, los datos requeridos, y sus restricciones,
entre otros factores que afecten su desarrollo, aunque no se entra
en los detalles de cada uno de estos factores. Y, por último, en la
tercera sección se definen los pormenores de los requisitos que el
usuario ha externado que el sistema actual cumple y por lo tanto el
nuevo sistema debe satisfacer.


\section{Descripción General}


\subsection{Perspectiva del producto}

Este sistema se relacionará con un software de facturación creando
archivos con los datos necesarios para generar las facturas. La interfaz
de facturación será provista por el software de facturación.

El sistema necesitará también contar con un SGBD externo.


\subsection{Funciones del producto}

El sistema permitirá a los clientes:
\begin{itemize}
\item Registrarse como usuarios
\item Realizar Encargos de comida a través de la web
\item Cancelar los encargos realizados
\end{itemize}
A los empleados:
\begin{itemize}
\item Gestionar los encargos de los clientes
\item Ver insumos faltantes y realizar pedidos a los proveedores
\item Gestionar los hornos
\item Gestionar los tipos de productos que los clientes pueden solicitar
\end{itemize}

\subsection{Características de los usuarios}

Se espera que el encargado de pedidos de la pizzería aprenda la funcionalidad
completa del software por lo que se espera que tenga conocimientos
básicos de informática además de estar muy familiarizado con los procesos
de la pizzería.


\subsection{Restricciones}

El sistema deberá utilizar para la facturación la interfaz provista
por el software actual de facturación de la pizzería.


\section{Requerimientos Específicos}


\subsection{Requerimientos Funcionales}

\selectlanguage{english}%
\begin{longtable}[l]{|l|l|>{\raggedright}m{0.7\linewidth}|}
\hline 
1 & Gestionar Encargo de Cliente & Recepciona pedido de comida de clientes y envia las ordenes necesarias
para satisfacerlo \tabularnewline
\hline 
2 & Seleccionar Horno & Elige en que horno se cocinara un determinado pedido\tabularnewline
\hline 
3 & Elegir Medio de Pago & Selecciona por que medio de pago abonara el cliente su encargo\tabularnewline
\hline 
4 & Definir Medio de Entrega & Selecciona la via de entrega del producto al cliente correspondiente\tabularnewline
\hline 
5 & Gestionar Pedido de Insumos & Confecciona la lista de insumos disponibles y faltantes y encarga
periodicamente su abasteciemiento\tabularnewline
\hline 
6 & Facturar Encargo & Proporciona al Sistema Factura los datos asociados a un encargo; incluyendo
cliente,comidas,bebidas y precios\tabularnewline
\hline 
7 & Registrar Cliente & Crea cuenta para un cliente en el sistema POS Pizzeria\tabularnewline
\hline 
8 & Ingresar al Sistema & Inicia cuenta de usuario en el sistema POS Pizzeria\tabularnewline
\hline 
9 & Realizar Encargo & Registra pedido de comida de un cliente\tabularnewline
\hline 
10 & Gestionar Menú & Añade,quita o modifica las comidas disponibles que puede encargar
el usuario\tabularnewline
\hline 
11 & Gestionar Horno & Añade y quita hornos del sistema y envia los encargo que debe realizar
cada uno\tabularnewline
\hline 
12 & Gestionar Módulos & Selecciona en que conjunto de partes los hornos cocinaran pedidos\tabularnewline
\hline 
13 & Seleccionar Política de Horno & Modifica formas de gestionar los hornos\tabularnewline
\hline 
14 & Añadir Horno & Agrega nuevo horno al sitema POS Pizzeria\tabularnewline
\hline 
15 & Crear Política de Horno & Agrega reglas para gestionar hornos\tabularnewline
\hline 
16 & Cancelar Encargo & Cancela un encargo\tabularnewline
\hline 
\end{longtable}

\selectlanguage{spanish}%

\subsection{Requerimientos de Interfaces Externas}

Los datos que requiere el software de facturación son:
\begin{itemize}
\item Código del cliente
\item Apellido del cliente
\item Nombre del cliente
\item Fecha-Hora de entrega
\item Tipo de entrega: local, mostrador o delivery
\item Monto Pagado
\item N° Cupón de tarjeta de crédito del pago
\item Lista de productos incluídos con código y cantidad de cada producto
\end{itemize}
\pagebreak{}


\section{Apéndice}


\subsection{Casos de Uso}

\includegraphics[scale=0.4]{\string"Caso de Uso de Sistema V2\string".png}


\subsubsection{Especificación de Casos de Uso}

\begin{tabular}{|>{\centering}p{0.2\columnwidth}|>{\centering}p{0.7\columnwidth}|}
\hline 
ID. CU & 9\tabularnewline
\hline 
Nombre & Realizar Encargo\tabularnewline
\hline 
Iniciador & Gestor de Encargos\tabularnewline
\hline 
Objetivo & Realizar un pedido de productos\tabularnewline
\hline 
Precondición & El gestor esta logueado en el sistema\tabularnewline
\hline 
Escenario principal de éxito & \begin{enumerate}
\item El gestor inicia pedido de encargo
\item El sistema muestra interfaz para seleccionar los productos a encargar
\item El cliente selecciona los productos
\item El cliente elige medio de pago
\item El sistema muestra el formulario de pago
\item El cliente completa el formulario de pago
\item El cliente confirma el envío del encargo
\item El sistema añade el encargo a la cola de encargos\end{enumerate}
\tabularnewline
\hline 
Postcondición & Se ha realizado un pedido y ha llegado al encargado de pedidos\tabularnewline
\hline 
\end{tabular}

\begin{tabular}{|>{\centering}p{0.8\columnwidth}|}
\hline 
Extensiones\tabularnewline
\hline 
\hline 
5.   El cliente no elige medio de pago\tabularnewline
\hline 
\begin{enumerate}
\item El cliente confirma el envío del encargo
\item El sistema añade el encargo a la cola de encargos\end{enumerate}
\tabularnewline
\hline 
\end{tabular}

\fbox{\begin{minipage}[t]{1\columnwidth}%
\includegraphics[scale=0.5]{\string"DA Realizar Encargo\string".png}%
\end{minipage}}

\textcompwordmark{}\linebreak{}


\begin{tabular}{|>{\centering}p{0.2\columnwidth}|>{\centering}p{0.7\columnwidth}|}
\hline 
ID. CU & 2\tabularnewline
\hline 
Nombre & Seleccionar horno\tabularnewline
\hline 
Iniciador & Encargado de pedidos\tabularnewline
\hline 
Objetivo & Elegir que horno cocinara un determinado encargo de un cliente\tabularnewline
\hline 
Precondición & Hay encargo/s en la cola de encargos y hay hornos disponibles\tabularnewline
\hline 
Escenario principal de éxito & \begin{enumerate}
\item El Encargado de Pedidos consulta hornos disponibles al Sistema.
\item El Sistema muestra los hornos disponibles.
\item El Encargado de Pedidos elige un horno libre.
\item El Encargado de Pedidos dispone encargo en horno seleccionado.
\item El Sistema valida operación.\end{enumerate}
\tabularnewline
\hline 
Postcondición & El encargo de cliente fue asignado a un horno\tabularnewline
\hline 
\end{tabular}

\textcompwordmark{}\linebreak{}


\fbox{\begin{minipage}[t]{1\columnwidth}%
\includegraphics[scale=0.5]{\string"DA Seleccionar Horno\string".png}%
\end{minipage}}

\includegraphics[scale=0.5]{\string"DSS Seleccionar Horno\string".png}

\begin{tabular}{|>{\centering}p{0.2\columnwidth}|>{\centering}p{0.7\columnwidth}|}
\hline 
ID. CU & 4\tabularnewline
\hline 
\hline 
Nombre & Definir medio de entrega\tabularnewline
\hline 
Iniciador & Encargado de Pedidos\tabularnewline
\hline 
Objetivo & Determinar por cual medio llegará el encargo al cliente\tabularnewline
\hline 
Precondición & Existe encargo del cliente\tabularnewline
\hline 
Escenario Principal de Éxito & \begin{enumerate}
\item El Encargado de Pedidos pregunta los distintos medios de envío al
Sistema
\item El Sistema muestra los medios de envío disponibles
\item El Encargado de Pedidos selecciona envío por delivery.
\item El Encargado de Pedidos registra empleado que realizara el envío.
\item El Sistema emite factura.
\item El Encargado de Pedidos registra pago que obtuvo el empleado del delivery.\end{enumerate}
\tabularnewline
\hline 
Postcondición & El medio por el cual se entregara el encargo fue definido\tabularnewline
\hline 
\end{tabular}

\begin{tabular}{|>{\centering}p{0.8\columnwidth}|}
\hline 
Extensiones\tabularnewline
\hline 
\hline 
3a. El Encargado de Pedidos selecciona medio entrega en el local
\begin{enumerate}
\item El Encargado de Pedidos registra pago que obtuvo del cliente.\end{enumerate}
\tabularnewline
\hline 
\end{tabular}

\begin{tabular}{|>{\centering}p{0.2\columnwidth}|>{\centering}p{0.7\columnwidth}|}
\hline 
ID. CU & 5\tabularnewline
\hline 
\hline 
Nombre & Realizar Pedido de Insumos\tabularnewline
\hline 
Iniciador & Encargado de Pedidos\tabularnewline
\hline 
Objetivo & Reponer el stock de insumos necesarios para los productos\tabularnewline
\hline 
Precondición & El punto de reposición de alguno/s de los insumos esta por debajo
del valor esperado\tabularnewline
\hline 
Escenario Principal de Éxito & \begin{enumerate}
\item El Sistema envía mensaje que avisa bajo stock de insumos
\item El Encargado de Pedidos consulta insumos de bajo stock
\item El Sistema muestra los insumos con stock por debajo del estimado
\item El Encargado de Pedidos selecciona insumo reponer, elige cantidad
y proveedor.
\item El Sistema envía pedido a Proveedor
\item El proveedor recibe pedido y lo valida\end{enumerate}
\tabularnewline
\hline 
Postcondición & El pedido de reposición de insumos fue efectuado satisfactoriamente.\tabularnewline
\hline 
\end{tabular}

\begin{tabular}{|>{\centering}p{0.8\columnwidth}|}
\hline 
Extensiones\tabularnewline
\hline 
\hline 
5a. El proveedor no puede acceder al sistema
\begin{enumerate}
\item El Sistema le notifica situación al Encargado de Pedidos
\item El Encargado de Pedidos realiza la operación de forma clasica\end{enumerate}
\tabularnewline
\hline 
\end{tabular}

\includegraphics[scale=0.5]{\string"DA Pedido de Insumos\string".png}

\includegraphics[scale=0.5]{\string"DSS Pedido Insumos\string".png}



\pagebreak{}

\tableofcontents{}
\end{document}
